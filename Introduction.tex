% !TEX root =big-fish.tex
\chapter{Introduction}
Performance pressure just keeps growing; to drive more sales and boost brand image, today�s Websites are increasingly dependent on sophisticated technologies that attract attention, hold interest or move visitors towards their virtual shopping cart. But if the technology behind the marketing vision creates delays or fails to work properly, visitors may quickly abandon the site and run to the competition.\\

The subject of this document is Big Fish, an online tool built on big data technology repurposed from Tacit Knowledge to monitor the availability and site performance of every major e-commerce site in the United States and the United Kingdom. It pulls and aggregates information from Alexa for traffic size, and Google for page speed to monitor over 2000 e-commerce sites.
\\

Big Fish is a 3D animation based on ThreeJS, a lightweight cross-browser JavaScript framework used to create and display animated 3D computer graphics on a web browser. It contrasts online store performance using a 3D graphical data view that allows to compare websites performance in a pleasant way, based on three metrics: response time, traffic and availability.
\\

The idea came as an attempt to break the new and distinguish Tacit Knowledge as an innovative brand. The concept consists of representing each website by a swimming fish in the ocean. The size of the fish represents traffic; its velocity represents response time; if the website is down the fish dies and floats into the surface.\\

One of the project challenges was to design a motion that doesn't repeat itself and looks as realistic as possible, while still representing data accurately. To that end 3D movement trajectories have been cautiously designed to allow perceiving websites basic parameters clearly, while tolerating collisions to happen so that a collision prediction correlated to a collision avoidance algorithms can operate to add an interactive, realistic touch to the motion.\\

This article serves as documentation for the motion design, as well as the collision prediction and collision avoidance algorithms.\\


\newpage